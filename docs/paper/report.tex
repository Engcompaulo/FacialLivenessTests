\documentclass[10pt,a4paper]{article}
\usepackage{times}
\usepackage{graphics}
\usepackage{graphicx}
\usepackage{natbib}
\usepackage{durhampaper}
\usepackage{subfigure}
% \usepackage{harvard}
% \usepackage[moderate]{savetrees}
\usepackage{url}

\title{Facial Liveness Testing: For The Web}
\author{} % leave; your name goes into \student{}
\student{Ryan Collins}
\supervisor{Prof A. Krokhin}
\degree{MEng Computer Science}

\date{\today}

\begin{document}

\maketitle

\begin{abstract}
% These instructions give you guidelines for preparing the final paper.  DO NOT change any settings, such as margins and font sizes.  Just use this as a template and modify the contents into your final paper.  Do not cite references in the abstract.

% The abstract must be a Structured Abstract with the headings {\bf Context/Background}, {\bf Aims}, {\bf Method}, {\bf Results}, and {\bf Conclusions}.  This section should not be longer than half of a page, and having no more than one or two sentences under each heading is advised.
\paragraph{Context/Background}
    TODO

\paragraph{Aims}
    TODO

\paragraph{Method}
    TODO

\paragraph{Results}
    TODO
\paragraph{Conclusions}
    TODO

\end{abstract}

\begin{keywords}
Facial liveness, convolutional neural networks, image quality metrics
\end{keywords}

\section{Introduction}
    % What is the project about?
    Currently, username and password authentication is commonplace throughout the web. However, username and password
    based authentication systems have a number of problems. Some common passwords can be broken using dictionary attacks,
    especially if they consist partially or entirely of a word in a standard dictionary. Furthermore, the process of shoulder surfing is possible (watching out
    for someone's password, and how they type it).

    An easy to use system is necessary to remove the choice from the user (in terms of password), relying on the user being automatically
    detected, and several confirmation methods to ensure the user is indeed who they say they are (and not just someone spoofing the system).
    Before such a system is developed, a facial liveness testing method must be found that operated in near real-time, and that is fairly accurate.

    % TODO: maybe remove this part, if we don't get that far.
    Furthermore, some communication scheme between a public device (e.g. client-side web browser code, that can be accessed), and the liveness system, must be present
    in order to ensure facial spoofing isn't present on the client input (reusing the same image).

\section{Related Work}
    TODO related Work
    % literature review esqe stuff.

\section{Solution}
    % Solution to the problem
\section{Results}
    % based on the solution, what results did we yield? What did we find out?
    TODO results
\section{Evaluation}
    % How well our system works, how well did it assess stuff, and how it can be improved.
    % what are the uses of our research? e.g. influence and improve the design of xyz...
    TODO evaluation

\section{Conclusions}
    % overall, what did hte projct show?
\bibliographystyle{dinat}
\bibliography{report}

\end{document}