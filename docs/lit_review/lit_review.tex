\documentclass[14pt]{article}
\usepackage[utf8]{inputenc}
\usepackage[english]{babel}
\usepackage{comment}
\usepackage{csquotes}
\title{Literature Review: Facial Authentication System for the Web}
\author{
        Ryan Collins\\
        Supervisor: Andrei Krokhin
}
\date{\today}

\usepackage[
  backend=biber,
  style=alphabetic,
  sorting=ynt
]{biblatex}
\addbibresource{lit_review.bib}
\begin{document}
\maketitle

\section{Introduction}
\paragraph{Problem Background}
Currently, username and password authentication is commonplace throughout the web. However, username and password
based authentication systems have a number of problems. Some common passwords can be broken using dictionary attacks,
especially if they are a word, or contain a word. Furthermore, the process of shoulder surfing is possible (watching out
for someone's password, and how they type it).

An easy to use system is necessary to remove the choice from the user (in terms of password), relying on the user being automatically
detected, and several confirmation methods to ensure the user is indeed who they say they are (and not just someone spoofing the system).

\paragraph{Areas of Research}
One initial consideration is to detect who a specific user is. Some existing measures require further images of a user
and require the training of a neural network. While this might work for a small number of faces, this isn't easily scalable.

With the given input, how can we be sure that it is original? How can we confirm that the input isn't spoofed,
and the person is in front of the camera at the correct time? The solution is to use liveness tests, to ensure
that the input is of someone who is alive, and not spoofed. 

One method is to carry out pupil tracking. This triggers an LED to light up, and tracks the user's pupil and
compares it to the location of the LED.
\cite{LivenessTestPupilTracking} For the web, while this method of using LEDs wouldn't necessarily match up,
one could instead use visual indicators within the web browser. 

Another method is to use eye blinking. The method proposed uses an ultra-sonic sensor,
which wouldn't be present for most devices connected to the web. However, alternative computer vision methods could be used.
Furthermore, this paper uses a blink pattern as a password, with the user being detected based on the dimensions of their eyes.\cite{LivenessTestBlinking}


%SAML2 for SSO cross origin (as effectively this is what our system will be using to tie into other sites)
As part of our system, once we carry out the necessary authentication, the system itself needs to tie-in well to
other existing web systems. SSO can be used, with the system acting as the single sign on manager. There are various
different methods available to do this, such as SAML2, OpenID and OAuth which are the most popular used on the web. 
The overall goal of SSO is to remove requiring separate IDs and passwords for many systems, instead relying on one single
set of credentials. The five key focuses of Identity and Access Management (and the overall goal of SSO) are:

\begin{enumerate}
  \item Authentication services: check which user is trying to log in. However, IAM suggests multiple methods of auth (rather than just one).
  \item Auth Management: what can a user access?
  \item Identity Management: Managing the accounts of the users using the service (creating their account, removing their account).
  \item Federated Identity: presents third parties with a token for authentication, rather than username/password. All passwords go through SSO portal.
  \item Compliance Management: Monitoring and reporting for audits. Check against security standards.
\end{enumerate}
\cite{SSOOverview}

In terms of security standards, a system for the web needs to follow the required standards to ensure it's fairly secure,
especially for the level of data that's required to be secured. For Identity Proofing (user provides evidence that they are indeed who they say they are),
there are three levels: IAL1 doesn't require any link to their real life identity, IAL2 has evidence that a real world identity exists and verifies that the link to this real world identity actually exists,
and IAL3 which requires a physical presence for identity proofing. For the authentication process, there are three levels: AAL1 requires either single or multi-factor authentication, and successful auth requires
the user to prove possession and control of the authenticator; AAL2 requires two methods of authentication (two factor); AAL3 requires the use of hardware authentication and two distinct methods of authentication.
\cite{NISTDigitalIdentityGuidelines}


SAML2 has also been researched into using a biometric system before - using fingerprints. This system
overcomes the initial security concerns with passwords, by using fingerprints (which can't be as easily guessed or
stolen). It was also found however that the client-side nature of such a system also lends itself well to further attacks,
due to the necessary transmission of raw biometric data between client and server. \cite{SAMLFingerPrint}
\section{Definitions}


\section{Important Issues of Identified Themes}
% This is the meat and potatoes. We need to answer each question proposed under the 'areas of research' section



\section{Proposed Direction of Project}
% Here is where we specify how we proceed with the project
Therefore, there is a need for a system that encorporates facial liveness, facial recognition and various other
extra security measures together, in a system that is secure for web-based authentication. By creating a service
accessible via an API, these system can be used both for web, as well as for IoT devices, which don't necessarily require
all the security measures of our system, but they're certainly welcome.
\section{Conclusion}
% The conclusion.
\label{references}
\printbibliography
\end{document}